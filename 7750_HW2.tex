%%%%%%%%%%%%%%%%%%%%%%%%%%%%%%%%%%%%%%%%%
% Programming/Coding Assignment
% LaTeX Template
%
% This template has been downloaded from:
% http://www.latextemplates.com
%
% Original author:
% Ted Pavlic (http://www.tedpavlic.com)
%
% Note:
% The \lipsum[#] commands throughout this template generate dummy text
% to fill the template out. These commands should all be removed when 
% writing assignment content.
%
% This template uses a Perl script as an example snippet of code, most other
% languages are also usable. Configure them in the "CODE INCLUSION 
% CONFIGURATION" section.
%
%%%%%%%%%%%%%%%%%%%%%%%%%%%%%%%%%%%%%%%%%

%----------------------------------------------------------------------------------------
%	PACKAGES AND OTHER DOCUMENT CONFIGURATIONS
%----------------------------------------------------------------------------------------

\documentclass{article}
%\usepackage{geometry}
%\geometry[letterpaper]
%\geometry{margin=2.5cm}
\usepackage{amsfonts}% Required for mathematical characters
\usepackage{fancyhdr} % Required for custom headers
\usepackage{lastpage} % Required to determine the last page for the footer
\usepackage{extramarks} % Required for headers and footers
\usepackage[usenames,dvipsnames]{color} % Required for custom colors
\usepackage{graphicx} % Required to insert images
\usepackage{listings} % Required for insertion of code
\usepackage{courier} % Required for the courier font
\usepackage{tikz} % Required for the Venn diagram

%\usepackage{lipsum} % Used for inserting dummy 'Lorem ipsum' text into the template

% Margins
\topmargin=-0.45in
\evensidemargin=0in
\oddsidemargin=0in
\textwidth=6.5in
\textheight=9.0in
\headsep=0.25in

\linespread{1.1} % Line spacing

% Set up the header and footer
\pagestyle{fancy}
\lhead{\hmwkAuthorName} % Top left header
\chead{\hmwkClass\ (\hmwkClassInstructor\ \hmwkClassTime): \hmwkTitle} % Top center head
\rhead{\firstxmark} % Top right header
\lfoot{\lastxmark} % Bottom left footer
\cfoot{} % Bottom center footer
\rfoot{Page\ \thepage\ of\ \protect\pageref{LastPage}} % Bottom right footer
\renewcommand\headrulewidth{0.4pt} % Size of the header rule
\renewcommand\footrulewidth{0.4pt} % Size of the footer rule

\setlength\parindent{0pt} % Removes all indentation from paragraphs

%----------------------------------------------------------------------------------------
%	CODE INCLUSION CONFIGURATION
%----------------------------------------------------------------------------------------

\definecolor{MyDarkGreen}{rgb}{0.0,0.4,0.0} % This is the color used for comments
\lstloadlanguages{Perl} % Load Perl syntax for listings, for a list of other languages supported see: ftp://ftp.tex.ac.uk/tex-archive/macros/latex/contrib/listings/listings.pdf
\lstset{language=Perl, % Use Perl in this example
        frame=single, % Single frame around code
        basicstyle=\small\ttfamily, % Use small true type font
        keywordstyle=[1]\color{Blue}\bf, % Perl functions bold and blue
        keywordstyle=[2]\color{Purple}, % Perl function arguments purple
        keywordstyle=[3]\color{Blue}\underbar, % Custom functions underlined and blue
        identifierstyle=, % Nothing special about identifiers                                         
        commentstyle=\usefont{T1}{pcr}{m}{sl}\color{MyDarkGreen}\small, % Comments small dark green courier font
        stringstyle=\color{Purple}, % Strings are purple
        showstringspaces=false, % Don't put marks in string spaces
        tabsize=5, % 5 spaces per tab
        %
        % Put standard Perl functions not included in the default language here
        morekeywords={rand},
        %
        % Put Perl function parameters here
        morekeywords=[2]{on, off, interp},
        %
        % Put user defined functions here
        morekeywords=[3]{test},
       	%
        morecomment=[l][\color{Blue}]{...}, % Line continuation (...) like blue comment
        numbers=left, % Line numbers on left
        firstnumber=1, % Line numbers start with line 1
        numberstyle=\tiny\color{Blue}, % Line numbers are blue and small
        stepnumber=5 % Line numbers go in steps of 5
}

% Creates a new command to include a perl script, the first parameter is the filename of the script (without .pl), the second parameter is the caption
\newcommand{\perlscript}[2]{
\begin{itemize}
\item[]\lstinputlisting[caption=#2,label=#1]{#1.pl}
\end{itemize}
}

%----------------------------------------------------------------------------------------
%	DOCUMENT STRUCTURE COMMANDS
%	Skip this unless you know what you're doing
%----------------------------------------------------------------------------------------

% Header and footer for when a page split occurs within a problem environment
\newcommand{\enterProblemHeader}[1]{
\nobreak\extramarks{#1}{#1 continued on next page\ldots}\nobreak
\nobreak\extramarks{#1 (continued)}{#1 continued on next page\ldots}\nobreak
}

% Header and footer for when a page split occurs between problem environments
\newcommand{\exitProblemHeader}[1]{
\nobreak\extramarks{#1 (continued)}{#1 continued on next page\ldots}\nobreak
\nobreak\extramarks{#1}{}\nobreak
}

\setcounter{secnumdepth}{0} % Removes default section numbers
\newcounter{homeworkProblemCounter} % Creates a counter to keep track of the number of problems

\newcommand{\homeworkProblemName}{}
\newenvironment{homeworkProblem}[1][Problem \arabic{homeworkProblemCounter}]{ % Makes a new environment called homeworkProblem which takes 1 argument (custom name) but the default is "Problem #"
\stepcounter{homeworkProblemCounter} % Increase counter for number of problems
\renewcommand{\homeworkProblemName}{#1} % Assign \homeworkProblemName the name of the problem
\section{\homeworkProblemName} % Make a section in the document with the custom problem count
\enterProblemHeader{\homeworkProblemName} % Header and footer within the environment
}{
\exitProblemHeader{\homeworkProblemName} % Header and footer after the environment
}

\newcommand{\problemAnswer}[1]{ % Defines the problem answer command with the content as the only argument
\noindent\framebox[\columnwidth][c]{\begin{minipage}{0.98\columnwidth}#1\end{minipage}} % Makes the box around the problem answer and puts the content inside
}

\newcommand{\homeworkSectionName}{}
\newenvironment{homeworkSection}[1]{ % New environment for sections within homework problems, takes 1 argument - the name of the section
\renewcommand{\homeworkSectionName}{#1} % Assign \homeworkSectionName to the name of the section from the environment argument
\subsection{\homeworkSectionName} % Make a subsection with the custom name of the subsection
\enterProblemHeader{\homeworkProblemName\ [\homeworkSectionName]} % Header and footer within the environment
}{
\enterProblemHeader{\homeworkProblemName} % Header and footer after the environment
}

%----------------------------------------------------------------------------------------
%	NAME AND CLASS SECTION
%----------------------------------------------------------------------------------------

\newcommand{\hmwkTitle}{Homework\ \#1} % Assignment title
\newcommand{\hmwkDueDate}{Thursday,\ September\ 12,\ 2013} % Due date
\newcommand{\hmwkClass}{STAT\ 7750} % Course/class
\newcommand{\hmwkClassTime}{2:00pm} % Class/lecture time
\newcommand{\hmwkClassInstructor}{Dr.\ Marco \ Ferreira} % Teacher/lecturer
\newcommand{\hmwkAuthorName}{Hongrui Zhang} % Your name

%----------------------------------------------------------------------------------------
%	TITLE PAGE
%----------------------------------------------------------------------------------------

\title{
\vspace{2in}
\textmd{\textbf{\hmwkClass:\ \hmwkTitle}}\\
\normalsize\vspace{0.1in}\small{Due\ on\ \hmwkDueDate}\\
\vspace{0.1in}\large{\textit{\hmwkClassInstructor\ \hmwkClassTime}}
\vspace{3in}
}

\author{\textbf{\hmwkAuthorName}}
\date{September 2, 2013} % Insert date here if you want it to appear below your name

%----------------------------------------------------------------------------------------

\begin{document}

\maketitle

%----------------------------------------------------------------------------------------
%	TABLE OF CONTENTS
%----------------------------------------------------------------------------------------

%%\setcounter{tocdepth}{1} % Unment this line if you don't want subsections listed in the ToC

%\newpage
%\tableofcontents
\newpage
%Chapter 1 exercises: 	25, 32, 37, 41, 46, 49, 61, 62.
%----------------------------------------------------------------------------------------
%	PROBLEM 1
%----------------------------------------------------------------------------------------

% To have just one problem per page, simply put a \clearpage after each problem

\begin{homeworkProblem}
\begin{enumerate}

%Listing \ref{homework_example} shows a Perl script.

%\perlscript{homework_example}{Sample Perl Script With Highlighting}

%\lipsum[1]

%\problemAnswer{
%\begin{center}
%\end{center}
Denote G="Good card", and P="Penalty card".
\begin{enumerate}
\item
As we know , 
\begin{center}
$S_A=\{G,G,G,P,P\}$.
\end{center}
Therefore, 
\begin{center}
 $P($\mathit{A}$ \ Good)=\frac{P(\{G,G,G\})}{P(S_A)}= \frac{3}{5}$
\end{center}
\item
For the reason that player $\mathit{B}$ chooses a card after player A. the all possible events of $\mathit{B}$ is:
\begin{center}
$S_B=\{G,G,P,P\}$
\end{center}
Therefore, the probability of "B Good" is:
\begin{center}
$P($\mathit{B}$ \ good| $\mathit{A}$ \ good)=\frac{2}{4}=\frac{1}{2}$
\end{center}
\item
Because A is "bad", all of the possible events of B is:
\begin{center}
$S_B=\{G,G,G,B\}$
\end{center}
Therefore, the "B bad" possibility is:
\begin{center}

$P(B \ good | A)=\frac{3}{4}$
\end{center}

\item

\end{enumerate}





%\begin{enumerate}
\item Two events $\mathit{A}$ and $\mathit{B}$ are independent if:
\begin{center}
$P(A \cap B) = P(A)*P(B)$
\end{center}
Obviously, \begin{center} $P(A \cap B) =0.031, and \ P(A)*P(B)=0.007242$
\end{center}
thus, $\mathit{A}$ and $\mathit{B}$ are called dependent.
\item
For a randomly selected Mexican American newborn, the probability that $\mathit{A}$ occurs is:
\begin{center}
$P(A)=0.142$
\end{center}
For a randomly selected Mexican American newborn, the probability that $\mathit{B}$ occurs is:
\begin{center}
$P(B)=0.051$
\end{center}
For a randomly selected Mexican American newborn, the probability that both $\mathit{A}$ and $\mathit{B}$ occurs is:
\begin{center}
$P(A \cup B)=0.162$
\end{center}
 \item
 The probability that event $\mathit{A}$ occurs given that event $\mathit{B}$ occurs is:
 \begin{center}
 $P(A|B)=\frac {P(A \cap B)}{P(B)}=\frac {0.031}{0.051} =0.254902$
 \end{center}
\end{enumerate}
\end{homeworkProblem}

%----------------------------------------------------------------------------------------
%	PROBLEM 2,2
%----------------------------------------------------------------------------------------

\begin{homeworkProblem}
PG exercise 6.10
\begin{enumerate}
\item
The probability that principal source of payment for given hospital discharge is the patient's private insurance is:
\begin{center}
$P(A)="payment from private insurance"=0.387$
\end{center}
\item
According to the Reference[20] "", 
\\
The probability that principal source of payment is Medicare, Medicaid, or some other government program is:
\begin{center}
$P("Payment \ from \ government \ programs") = P("Medicare")+P("Medicaid")+P("Other \ govt. \ program")=0.345+0.116+0.033=0.494$
\end{center}
\item
According to the Reference[20], public programs include Medicare, Medicaid, Workers' Compensation, and other government programs. Given that the principal source of payment is a government program, the probability that it is Medicine is:
\begin{center}
$P("Medicare" | "Payment \ from \ government\ programs")$\\
$= \frac{"Medicare"}{"Payment \ from \ government \ programs"}=\frac{0.345}{0.494}=0.698$
\end{center}
\end{enumerate}
%/lipsum [3-5]

\end{homeworkProblem}
\begin{homeworkProblem}
PG exercise 6.11
\begin{enumerate}
\item
Considering the 47-year-old woman and the 59-year-old man from this population are unrelated, the probability that both individuals are uninsured is equal and independent. It is:
\begin{center}
$P(A' \cap B')= P(A')*P(B')=0.123*0.123=0.015$
\end{center}
\item
The probability that both adults have health insurance is:
\begin{center}
$P(A \cap B) =  P(A)*P(B)=(1-0.123)*(1-0.123)=0.769$
\end{center}
\item
For the 5 unrelated 45-64 year-old adults, the probability that all five are uninsured is:
\begin{center}
$P(A' \cap B' \cap C' \cap D' \cap E')=P(A') \cap P(B') \cap P(C') \cap P(D') \cap P(E') =0.123^5=2.815*10^{-5} $
\end{center}
\end{enumerate}
\end{homeworkProblem}
%----------------------------------------------------------------------------------------

\end{document}
